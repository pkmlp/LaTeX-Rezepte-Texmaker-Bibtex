
In der Präambel:

\begin{verbatim}

% für mathematische Symbole und Formeln
\usepackage{amsmath, amssymb}

\end{verbatim}

\tcblower

Im Dokument: 

\begin{verbatim}

Pythagoras sagt: Seien $a$ und $b$ die Katheten und $c$ die
Hypotenuse, dann gilt: 

\begin{displaymath}
  a^2+b^2=c^2 
\end{displaymath}
 
Somit gilt für die Hypothenuse: 

\begin{displaymath}
  c=\sqrt{a^2+b^2} 
\end{displaymath}

\bigskip 
... oder auch ...
\bigskip 
\bigskip 

Für
\begin{displaymath}
a,b \in \mathbb{R}
\end{displaymath}

gilt: 
\begin{displaymath}
\begin{split}
(a+b)^{2} = a^{2} + 2ab + b^{2} \\
(a-b)^{2} = a^{2} - 2ab + b^{2} \\
(a-b)(a+b) = a^{2} - b^{2}
\end{split}
\end{displaymath}


\end{verbatim}
