
Für Sätze, Lemmata und so weiter stellt LaTeX eine generische Theoremumgebung zur Verfügung. Man kann sich nach seinem Gutdünken Umgebungen zusammenbasteln. Hier einige Beispiele:

\begin{satz} 
   Ein nummerierter Satz. 
\end{satz}

\begin{proof}
   Den wir hier gleich auch mit der Standard-Umgebung proof Beweisen!
\end{proof}

Die Sätze werden automatisch numeriert. Sollen die Sätz in einem Abschnitt die entsprechende Kapitelnummer haben, so sieht das wie folgt aus:

\begin{theorem} 
   Ein nummeriertes Theorem, das die Kapitelnummer enthält. 
\end{theorem}

\begin{proof}
   Das wir hier gleich auch mit der Standard-Umgebung proof Beweisen!
\end{proof}

Theoreme auch ganz ohne Nummer sind möglich:

\begin{behauptung}
   Eine Behauptung
\end{behauptung}

\begin{proof}
   Die wir hier gleich auch mit der Standard-Umgebung proof Beweisen!
\end{proof}
