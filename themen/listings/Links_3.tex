
Auch wenn Sie nicht Informatiker sind, möchten Sie in Ihrem Dokument vielleicht Quell-Code zeigen.

\begin{leftbar} 
Dies ist ein weiterer Anwendungsfall, der die Überlegenheit von \LaTeX\ gegenüber WYSIWYG Word Prozessoren aufzeigt. Das Programm-Listing wird aus der Quell-Datei eingelesen und kann mit \LaTeX\ betr. Syntax-Coloring und Syntax-Highlighting der gewohnten Entwicklungsumgebung (IDE) angepasst werden. Damit sehen Programm-Listings im \LaTeX-Dokument aus wie in der gewohnten Entwicklungsumgebung.  
\end{leftbar}

Das Package verfügt über viele weitere Einstellungen zur Darstellung des Quell-Codes. Die Nummerierung der Zeilen ermöglicht zudem, im Fliesstext konkrete Programmzeilen zu referenzieren.
