
Das Quellen-/ Literaturverzeichnis soll auf einer ungeraden (rechten) Seite des Dokumentes erscheinen.


\textbf{Zur Erinnerung:} Die Erstellung des Quellen-/ Literaturverzeichnis erfodert ein spezielles \LaTeX-Werkzeug (bibLaTex). Damit alles korrekt erstellt wird, muss folgende Befehls-Sequenz eingehalten werden\footnote{Befehle und Funktionstasten wie in TeXmaker verwendet}:

\begin{enumerate}
\setlength\itemsep{-1em}
\item PDFLaTeX (F6)
\item PDFLaTeX (F6)
\item BibTeX (F11 - nur wenn Literaturverzeichnis vorhanden)
\item MakeIndex (F12 - nur wenn Stichwortverzeichnis vorhanden)
\item PDFLaTeX (F6)
\item PDFLaTeX (F6)
\item PDF ansehen (F7)
\end{enumerate}

\textbf{Zur Erinnerung:} In diesem Beispiel wird mit BibTeX gearbeitet, nicht mit Biber. Bei Editoren oder IDEs die mit biber als Backend für die Literaturverwaltung arbeiten sind andere \LaTeX-Befehle und \LaTeX-Packages notwendig.

\textbf{Hinweis:} In den meisten IDEs kann obige Befehls-Sequenz eingestellt werden und über "'schnelles Übersetzen"' mit einem Mausklick ausgeführt werden.
