
In der Präambel:

\begin{verbatim}

% Package für Abkürzungsverzeichnis mit oder ohne Angabe der Seite
\usepackage[printonlyused]{acronym}
%\usepackage[printonlyused,withpage]{acronym}

\end{verbatim}

\tcblower

Erfassen der Abkürzungen in einer separaten Datei:

\begin{verbatim}

\acro{eth}[ETH]{Eidgenössische Technische Hochschule}
\acro{ethz}[ETHZ]{Eidgenössische Technische Hochschule Zürich}
\acro{epfl}[EPFL]{École polytechnique fédérale de Lausanne}
\acro{id}[ID]{Informatikdienste}
\acro{pm}[PM]{Portfolio Management}
\acro{ppf}[PPF]{Procurement and Portfolio Management}

\end{verbatim}


Im Dokument: 

\begin{verbatim}

Mein Arbeitgeber ist die \ac{eth}, genauer die \ac{ethz}. Neben 
der \ac{ethz} gibt es auch noch eine \acs{eth} in der Westschweiz, 
nämlich die \ac{epfl}. An der \acs{eth} bin ich bei den \ac{id} im 
Bereich \ac{pm} tätig. Das \ac{pm} ist eine Gruppe innerhalb der 
Sektion \ac{ppf}.

\end{verbatim}
